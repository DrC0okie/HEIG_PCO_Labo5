%! suppress = MissingLabel
\documentclass{article}
\usepackage[T1]{fontenc}
\usepackage[main=english]{babel}
\usepackage{url}
\usepackage{lastpage}
\usepackage{fancyhdr}
\usepackage{graphicx}
\usepackage[a4paper, margin=2cm, footskip=18.3pt]{geometry}
\usepackage{listings}
\usepackage[usenames]{color}

\newcommand{\header} {
    \setlength{\headheight}{30pt}\pagestyle{fancy}
    \fancyhead[L]{\includegraphics[height=20pt]{~/Templates/heig-logo}}\fancyhead[C]{PCO 2023\\ Lab 5}
    \fancyhead[R]{Timothée Van Hove \& Aubry Mangold\\\today}\fancyfoot[C]{}
    \fancyfoot[R]{Page \thepage~sur \pageref{LastPage}}\renewcommand{\footrulewidth}{0.3pt}
}


\definecolor{mygreen}{rgb}{0,0.6,0}
\definecolor{mygray}{rgb}{0.5,0.5,0.5}
\definecolor{mymauve}{rgb}{0.58,0,0.82}

\lstset{frame=tb,
    language=C++,
    aboveskip=3mm,
    belowskip=3mm,
    showstringspaces=false,
    columns=flexible,
    basicstyle={\small\ttfamily},
    numbers=left,
    numberstyle=\tiny\color{mygray},
    keywordstyle=\color{blue},
    commentstyle=\color{mygreen},
    stringstyle=\color{mymauve},
    breaklines=true,
    breakatwhitespace=true,
    tabsize=4
}

\begin{document}
    \header


    \section{Introduction}


    \section{Analysis}


    \section{Conception}

    % TODO 1-2 explicative sentences by Tim

    \section{Implementation}

    % Listing
    \begin{lstlisting}[caption={The shared section access routine.}, captionpos=b, label=lst:1]
    \end{lstlisting}

    \section{Automated Testing of Barber and Clients Sequences}

    To ensure the correctness of our simulation, we implemented automated tests targeting the behavioral sequences of both the barber and the clients. These tests are crucial for verifying that the sequences adhere to predefined, valid patterns that reflect the intended logic of our program.

    \subsection{Defining Valid Sequences}

    Our testing strategy involves defining "valid sequences" for both the barber and the clients. These sequences are essentially templates of actions that the barber and clients are expected to perform in a specific order under normal operating conditions.

    These valid sequences are defined as arrays of enums, representing distinct actions. By using enums, we provide a readable representation of each action in the sequences.

    \subsection{Recording Sequences During Execution}

    As the simulation runs, each action taken by the barber or a client is recorded into a vector. This is accomplished by placing "push back" operations within the PcoSalon class's methods, which capture the sequence of actions as they occur in real-time. These vectors serve as the actual sequences to be tested against our predefined valid sequences at the end of the simulation.

    \subsection{Analyzing and Validating the Sequences}

    Once the simulation ends, the testing phase begins. Here, we analyze the recorded sequences to ensure they align with our predefined valid sequences. The analysis involves segmenting the sequences and then systematically comparing each segment against the valid sequences.

    For the barber, whose sequences can vary based on situational factors (such as whether the barber was asleep or awake at the start of the sequence), the segmentation is based on the consistent ending action in each cycle. For clients, whose actions form repeatable loops, the segmentation is based on the start of each loop.

    If a recorded sequence matches any of the valid sequences, it is deemed correct. If not, the sequence is flagged as incorrect, indicating a potential issue in the simulation logic or sequence implementation.


    \section{Manual tests}

    The following tests have been executed both on our local machines and on the reds VM. These are manual tests, as we did not write automated ones for those cases

    Synchronization tests :
    \begin{itemize}
        \item Plus de clients(8) que de sièges(2)
        \item Plus de sièges(8) que de clients(2)
        \item Le même nombre de clients que de sièges(4)
        \item Pas de siege et un seul client
        \item Pas de siege et plusieurs clients (8)
        \item Pas de siege et pas de client
        \item Plusieurs sièges(4) et un seul client
        \item Un seul siège et plusieurs clients(20)
        \item Beaucoup de clients (30) et de sièges(20)
    \end{itemize}

    Program end tests :
    \begin{itemize}
        \item Un client entre dans le salon
        \item Un client va s'asseoir
        \item Un client se fait couper les cheveux
        \item Un client se fait couper les cheveux et personne n'attend
        \item Un client attend que ses cheveux poussent
        \item Un client fait un tour
        \item Le barbier dort
        \item Le barbier se réveille
    \end{itemize}

    Client entrance order tests :
    \begin{itemize}
        \item Lorqu'il y a un seul client
        \item Lorsqu'il y a plusieurs clients(8)
    \end{itemize}

    Hair cut order test:
    \begin{itemize}
        \item Lorsqu'il y a plusieurs clients(20) et sièges (10)
    \end{itemize}

    Other tests :
    \begin{itemize}
        \item Les clients ne font un tour que s'il n'y a plus de place dans le salon
        \item Les clients ne rentrent chez eux que si le salon est fermé
        \item Le barbier va se coucher quand il n'y a pas de clients
        \item Lorsque le barbier dort et qu'un client est entré dans le salon, il le réveille
        \item Le barbier s'arrête quand il n'y a plus de clients et que la journée est terminée
    \end{itemize}

    \subsection{Edge case}
    We discovered an edge case that we had to test for. Most of the time, when the simulation begins, the barber goes to sleep as there are no client in the salon yet. However, in rare scenarios, the barber is awake when the first client comes into the salon. In this case, we had to make the barber wait until the client is fully inside the salon (wait that the animation is finished) before picking the next client, otherwise, the barber would pick the nex ticket, while no client have already taken a ticket, leading to a deadlock. To test for this case, we simply added a sleep(1) before the main while loop in the barber::run method.
    Note that when this edge case arrise, we have chosen to put the first client directly on the barber chair, as we found that making the client in the waiting chair useless. In addition, this choice allows an other client to enter the salon.

    \section{Conclusion}

    \section*{Appendix}

    \section{Appendix name}
    \begin{figure}[!htb]
        \centering
        \includegraphics[width=0.6\textwidth]{./doc/assets/}
        \caption{Train route 1}
    \end{figure}

\end{document}
