%! suppress = MissingLabel
\documentclass{article}
\usepackage[T1]{fontenc}
\usepackage[main=english]{babel}
\usepackage{url}
\usepackage{lastpage}
\usepackage{fancyhdr}
\usepackage{graphicx}
\usepackage[a4paper, margin=2cm, footskip=18.3pt]{geometry}
\usepackage{listings}
\usepackage[usenames]{color}

\newcommand{\header} {
    \setlength{\headheight}{30pt}\pagestyle{fancy}
    \fancyhead[L]{\includegraphics[height=20pt]{~/Templates/heig-logo}}\fancyhead[C]{PCO 2023\\ Lab 5}
    \fancyhead[R]{Timothée Van Hove \& Aubry Mangold\\\today}\fancyfoot[C]{}
    \fancyfoot[R]{Page \thepage~sur \pageref{LastPage}}\renewcommand{\footrulewidth}{0.3pt}
}


\definecolor{mygreen}{rgb}{0,0.6,0}
\definecolor{mygray}{rgb}{0.5,0.5,0.5}
\definecolor{mymauve}{rgb}{0.58,0,0.82}

\lstset{frame=tb,
    language=C++,
    aboveskip=3mm,
    belowskip=3mm,
    showstringspaces=false,
    columns=flexible,
    basicstyle={\small\ttfamily},
    numbers=left,
    numberstyle=\tiny\color{mygray},
    keywordstyle=\color{blue},
    commentstyle=\color{mygreen},
    stringstyle=\color{mymauve},
    breaklines=true,
    breakatwhitespace=true,
    tabsize=4
}

\begin{document}
    \header


    \section{Introduction}


    \section{Analysis}


    \section{Conception}

    % TODO 1-2 explicative sentences by Tim

    \section{Implementation}

    % Listing
    \begin{lstlisting}[caption={The shared section access routine.}, captionpos=b, label=lst:1]
    \end{lstlisting}


    \section{Tests}

    Cas testés pour la synchronistation :
    \begin{itemize}
        \item Plus de clients(8) que de sièges(2)
        \item Plus de sièges(8) que de clients(2)
        \item Le même nombre de clients que de sièges(4)
        \item Pas de siege et un seul client
        \item Pas de siege et plusieurs clients (8)
        \item Pas de siege et pas de client
        \item Plusieurs sièges et un seul client
        \item Un seul siège et plusieurs clients(8)
        \item Beaucoup de clients (30) et de suèges(20)
    \end{itemize}

    Terminaison du programme pendant les animations suivantes :
    \begin{itemize}
        \item Un client entre dans le salon
        \item Un client va s'asseoir
        \item Un client se fait couper les cheveux
        \item Un client se fait couper les cheveux et personne n'attend
        \item Un client attend que ses cheveux poussent
        \item Un client fait un tour
        \item Le barbier dort
        \item Le barbier se réveille
    \end{itemize}

    L'ordre d'entrée des clients est bien respecté :
    \begin{itemize}
        \item Lorqu'il y a un seul client
        \item Lorsqu'il y a plusieurs clients
    \end{itemize}

    Autres tests :
    \begin{itemize}
        \item Les clients ne font un tour que s'il n'y a plus de place dans le salon
        \item Les clients ne rentrent chez eux que si le salon est fermé
        \item Le barbier va se coucher quand il n'y a pas de clients
        \item Lorsque le barbier dort et qu'un client est entré dans le salon, il le réveille
        \item Le barbier s'arrête quand il n'y a plus de clients et que la journée est terminée
    \end{itemize}

    \section{Conclusion}

    \section*{Appendix}

    \section{Appendix name}
    \begin{figure}[!htb]
        \centering
        \includegraphics[width=0.6\textwidth]{./doc/assets/}
        \caption{Train route 1}
    \end{figure}

\end{document}
